
%- description of the LASERII DAQ; interface with ATLAS; type of runs, data sent (fragments structure) ;

The \lasii{} system is controlled by the \atlas{} DAQ through the {\tt TileLaserRCD} application,
which includes the \lasmodule{} library. This library has three main functions:
\begin{itemize}
\item control of the \lascar{} \vme{} board, in order to configure the \lasii{} system
hardware at the start of a new run and at its end;
\item monitoring of the \lasii{} system behaviour during a run in order to take actions in
case of problems;
\item publication in the Information Service (IS) of the status of the \lasii{} system.
\end{itemize}

\subsubsection{Control and monitoring of the system}
The \lasmodule{} library controls and monitors the \lasii{} system by communicating with the
\lascar{} board, thanks to the fact that the {\tt TileLaserRCD} application runs on the
Single Board Computer (sbc-til-ttc-las-01) that controls the \lasii{} \vme{} crate.

\paragraph{\lascar~configuration}
~~\\
At the start of a new \atlas{} or \tilecal{} run, the \lasmodule{} performs the following
actions:
\begin{enumerate}
\item It checks the state of \lascar{} and reports a fatal error in case it is not found,
\item It retrieves the run and configuration parameters from the IS, in particular
from the "TileParams.LasIIConfig" parameters (see Table~\ref{tab:IS:lasIIconfig}) in order to configure the \lasii{}
system according to them,
\item It sets up the \lascar{} module according to the run type, as explained
in the following paragraphs.
\end{enumerate}

\begin{table}[ht]
  \begin{center}
    \caption{Parameters of the "TileParams.LasIIConfig" object in IS.}\label{tab:IS:lasIIconfig}
    \begin{tabular}{lll}
      \hline
      Parameter & Content & Type \\
      \hline
      Filter & Filter wheel position (from 1 to 8) & u32 \\
      ShutterOpen & True means that the shutter is open & bool \\
      Intensity & \las{} intensity value in arbitrary unit & u32 \\
      \hline
    \end{tabular}
  \end{center}
\end{table}

\paragraph{Running modes}

\begin{itemize}
\item{LaserCalib or LaserAlpha}~\\
This type of run is needed to perform a calibration of the \lasii{} system. In this
LaserCalib mode, \lascar{} controls the \lasii{} components, without sending any
\las{} pulse to \tilecal{}. Amplitudes of the signals produced by the photodiodes
and the photomultipliers of the \lasii{} system are sent to the \atlas{} DAQ by \lascar{}.
At the start of run, \lascar{} is put in the pedestal mode, i.e. the amplitudes
are digitized randomly. Once enough events have been recorded in this mode,
\lascar{} is switched to the next calibration mode, starting from the alpha mode, then the
LED mode and finally the linearity mode increasing the injected charge from 0 to 60000 DAC counts by steps
of 10000. After the last linearity point, the run is either stopped (in the case of the
LaserCalib run type) or \lascar{} is switched back to the alpha mode until the run is stopped
manually (in the case of the LaserAlpha run type). The short data fragment (Figure \ref{fig:shortfrag} in \ref{app}) is transmitted in this mode.


\item{Laser}~\\
This is the main running mode to calibrate \tilecal{}, when the \las{} pulses are sent to the
calorimeter under the control of the SHAre Few Trigger (\shaft) module. Amplitudes of the signals produced by the 
photodiodes and the photomultipliers of the \lasii{} system are sent to the \atlas{} DAQ by 
\lascar{}. At the start of run, the \lasmodule{} library configures \lascar{} to move the filter
wheel to the requested position, as well as the shutter, and to start the \las{} pump with the 
required intensity. Then it starts \lascar{} in the Laser mode. The long data fragment (see Figures  \ref{fig:longfraga}-\ref{fig:longfragc} in \ref{app}) is transmitted in this mode.

\item{Physics}~\\
During physics runs, the \lasii{} system sends \las{} pulses to \tilecal{}, exactly like in
a Laser run described previously. The only difference is at the start of run. In the case of
a physics run, \lascar{} is first put in the internal calibration mode. This mode performs
similar internal calibrations as in the LaserCalib runs (pedestal, alpha and LED, but no
linearity) except that an internal trigger is used instead of the external signal coming from
\shaft{}, and that the amplitudes are not sent to the \atlas{} DAQ but rather stored in the 
\lascar{} memory. As soon as this internal calibration is completed, the \lasmodule{} library
sets \lascar{} in Laser mode and fills histograms with the amplitudes recorded during this
calibration. In order to save time, the filter wheel, the shutter and the \las{} pump
are set to the required states prior to the start of the internal calibration. The long data fragment (see Figures \ref{fig:longfraga}-\ref{fig:longfragc} in \ref{app}) is transmitted in this mode.

\item{Other types of runs}~\\
For all other types of runs, \lascar{} is started in idle mode, but reacting to the
trigger by sending fragments that contain only the header and the trailer.

\paragraph{Actions during a run}
~~\\
Periodically during a run, the \lasmodule{} library performs some actions:
\begin{itemize}
\item It checks the state of the \lascar{} TTCrx circuit. In case of fault, an error is reported.
\item It checks the state of the busy counter. If \lascar{} has been busy for a period longer
than the given limit (in principle 5~s), a stopless removal request is sent to the DAQ.
\item If \lascar{} is in Laser mode, it checks several parameters:
  \begin{itemize}
  \item The position of the filter wheel and of the shutter. In case the actual position does not
    correspond to the requested one, an error is reported and \lascar{} is asked to move the
    component to the requested position.
  \item The \las{} pump status. In case the pump is off, an error is reported and \lascar{} is
    asked to switch it on.
  \item The number of \las{} pulses that have been 
    requested, emitted and recorded. In case of missing pulses, warnings are reported.
  \item The \lascar{} watchdog is restarted. If after five minutes the watchdog has
    not been restarted, \lascar{} will automatically stop the \las{} pump. This allows protecting
    the \lasii{} system from crashes of the \atlas{} DAQ.
  \end{itemize}
\end{itemize}

\paragraph{Stopless removal mechanism}
~~\\
During a run, if the \atlas{} DAQ decides to disable the \lasmodule{}, the shutter is closed,
the \las{} pump is switched off and \lascar{} is stopped. The busy channel of the corresponding
Local Trigger Processor (LTP) is also disabled. An SMS is sent to the DAQ expert mobile phone.

\paragraph{End of run}
At the end of any run, the shutter is closed, the \las{} pump is switched off and \lascar{} is
stopped.

\subsubsection{Information published}
The \lasmodule{} library publishes periodically several parameters of the \lasii{} system,
for monitoring purposes. These parameters are published to the IS in the "TileParams.LasIIStatus" 
object, described in Table~\ref{tab:IS:lasIIstatus} (\ref{app}).




