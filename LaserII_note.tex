%-------------------------------------------------------------------------------
% This file provides a skeleton ATLAS document.
%-------------------------------------------------------------------------------
% Specify where ATLAS LaTeX style files can be found.
\newcommand*{\ATLASLATEXPATH}{latex/}
% Use this variant if the files are in a central location, e.g. $HOME/texmf.
% \newcommand*{\ATLASLATEXPATH}{}
%-------------------------------------------------------------------------------
\documentclass[UKenglish,texlive=2013]{\ATLASLATEXPATH atlasdoc}
% The following command is needed by arXiv to ensure use of pdflatex.
% It should be included in the first 5 lines of the preamble.
% \pdfoutput=1
% The language of the document must be set: usually UKenglish or USenglish.
% british and american also work!
% Commonly used options:
%  texlive=YYYY          Specify TeX Live version (2013 is default).
%  atlasstyle=true|false Use ATLAS style for document (default).
%  coverpage             Create ATLAS draft cover page for collaboration circulation.
%                        See atlas-draft-cover.tex for a list of variables that should be defined.
%  cernpreprint          Create front page for a CERN preprint.
%                        See atlas-preprint-cover.tex for a list of variables that should be defined.
%  PAPER                 The document is an ATLAS paper (draft).
%  CONF                  The document is a CONF note (draft).
%  PUB                   The document is a PUB note (draft).
%  txfonts=true|false    Use txfonts rather than the default newtx - needed for arXiv submission.
%  paper=a4|letter       Set paper size to A4 (default) or letter.

%-------------------------------------------------------------------------------
% Extra packages:
\usepackage{\ATLASLATEXPATH atlaspackage}
% Commonly used options:
%biblatex=false|true   Use biblatex (default) or bibtex for the bibliography.
%  backend=biber         Use the biber backend rather than bibtex.
%  subfigure|subfig|subcaption  to use one of these packages for figures in figures.
%  minimal               Minimal set of packages.
%  default               Standard set of packages.
%  full                  Full set of packages.
%-------------------------------------------------------------------------------
% Style file with biblatex options for ATLAS documents.
\usepackage{\ATLASLATEXPATH atlasbiblatex}

% Package for creating list of authors and contributors to the analysis.
%\usepackage{\ATLASLATEXPATH atlascontribute}

% Useful macros
\usepackage{\ATLASLATEXPATH atlasphysics}
\usepackage{comment}
% See doc/atlas-physics.pdf for a list of the defined symbols.
% Default options are:
%   true:  journal, misc, particle, unit, xref
%   false: BSM, hion, math, process, other, texmf
% See the package for details on the options.

% Files with references for use with biblatex.
% Note that biber gives an error if it finds empty bib files.
\addbibresource{LaserII_note.bib}
%\addbibresource{bibtex/bib/ATLAS.bib}

% Paths for figures - do not forget the / at the end of the directory name.
\graphicspath{{logos/}{figures/}}

% Add you own definitions here (file LaserII_note-defs.sty).
%\usepackage{LaserII_note-defs}
\newcommand{\laser}{laser}
\newcommand{\las}{laser}
\newcommand{\lasa}{{\sc LASER~I}}
\newcommand{\lasi}{LaserI}
\newcommand{\lasii}{LaserII}
\newcommand{\lasb}{Laser175}
\newcommand{\atlas}{{\sc ATLAS}}
\newcommand{\pmts}{{\sc PMTs}}
\newcommand{\pmt}{{\sc PMT}}
\newcommand{\tilecal}{TileCal}
\newcommand{\vme}{{\sc VME}}
\newcommand{\phocal}{{\sc PHOCAL}}
\newcommand{\licphd}{LicPhD}
\newcommand{\licpmt}{LicPMT}
\newcommand{\licmot}{LicMot}
\newcommand{\lascar}{{\sc LASCAR}}
\newcommand{\polas}{{\sc POLAS}}
\newcommand{\charinjsplit}{{\sc CIS}}
\newcommand{\coimbra}{Coimbra}
\newcommand{\mum}{$\mu m$}

\newcommand{\shaft}{SHAFT}
\newcommand{\lasmodule}{{\tt TileLasIIModule}}


%-------------------------------------------------------------------------------
% Generic document information
%-------------------------------------------------------------------------------

% Title, abstract and document 
\input{LASERII_note-metadata}
% Author and title for the PDF file
\hypersetup{pdftitle={Upgrade of the Laser calibration system of the ATLAS Tile Calorimeter},pdfauthor={The LaserII group}}

\linenumbers

%-------------------------------------------------------------------------------
% Content
%-------------------------------------------------------------------------------
\begin{document}

\renewcommand\appendix{\par
  \setcounter{section}{0}
  \setcounter{subsection}{0}
  \setcounter{figure}{0}
  \setcounter{table}{0}
  \renewcommand\thesection{Appendix} %\Alph{section}}
  \renewcommand\thefigure{\Alph{section}\arabic{figure}}
  \renewcommand\thetable{\Alph{section}\arabic{table}}
}

\maketitle

\tableofcontents

% List of contributors - print here or after the Bibliography.
%\PrintAtlasContribute{0.30}
%\clearpage

%-------------------------------------------------------------------------------
\section{Introduction}
\label{sec:intro}
%-------------------------------------------------------------------------------

The Atlas Tilecal hadronic calorimeter (TileCal) \cite{ref:tilecal} is equipped with a 3-level calibration system to monitor and  calibrate the response of the active devices (plastic scintillating tiles), the response of the PMTs receiving the light from clusters of tiles (cells), and the response of the Front End and digitization electronics of the individual readout channels. \par
The global detector calibration is performed monthly by measuring the individual cell response to a reference excitation produced by calibrated Cesium sources \cite{ref:cesiumscan} which float inside the detector through a suitable pipe network.
Calibration of the PMTs and of the readout chain are performed between two sucessive Cesium scans; laser pulses are distributed to each individual PMT photocathode (approximatively 10,000 channels) by a suitable optical distribution line having a powerful laser (several $\mu$J per pulse) as a source. Laser calibrations can be performed every two-three days during pauses of LHC collisions and during collision runs by pulsing the laser in the empty bunch intervals.  \par
During the Long Shutdown I of LHC operations, the Tilecal \laser~calibration system has been re-designed, tested, and installed for the Run II operations. The shortcomings of the optical part have been studied, understood and solved. The \laser~light injected in the PMTs is measured by sets of photodiodes at several stages of the optical path. The monitoring of the photodiodes is performed by a redundant internal calibration scheme using an LED, a radioactive source, and a charge injection system. \par
A challenge for the new \laser~project was lying in the design of electronics boards that would overcome the shortcomings of the previous \laser~system (interface with the LHC clock and saturation of the electronics for higher \laser~intensities) and that would be compatible with the new requirements (increase of the number of photodiodes, new internal calibration scheme). A new electronics has been designed to achieve these goals.\par
This document proposes a detailed description of the \laser~calibration system of the TileCal and is organized as follows. In a first part the \lasi~system, in operation up to October 2014, is depicted, its performance quoted and its shortcomings listed. The new \laser~structure, dubbed \lasii, is described in a second part. Its main components (electronics, optics, data acquisition and control system) are specified. It should be noticed that this note aims at summarizing the hardware side of the \lasii~system. A separated document related to the performance will follow. 


%-------------------------------------------------------------------------------
\section{The \lasi~system during P1: performance and shortcomings}
\label{sec:detector}
%-------------------------------------------------------------------------------

\input LaserI.tex

%-------------------------------------------------------------------------------
\section{The \lasii~system}
\label{sec:result}
%-------------------------------------------------------------------------------

\subsection{Overview}

\input laserII_overview.tex

\subsection{Electronics}

\subsubsection{Photodiode box}
\label{photodiodebox}

\input photodiodebox.tex

\subsubsection{PMT box}
\label{pmtbox}

\input pmtbox.tex

\subsubsection{Internal calibration system: PHOCAL}

\input phocal.tex

\subsubsection{VME SBC}

\input sbc.tex

\subsubsection{LASCAR}

\input lascar.tex

\subsubsection{POLAS}

\input polas.tex

\subsection{Optics}

\subsubsection{Optics box}

\input opticsbox.tex

\subsubsection{Optical filters}

\input opticalfilters.tex

%\subsubsection{Optical path}
%\subsubsection{Light mixers and filter wheel}
%\subsubsection{Beam expander}

\subsection{DAQ}
\label{sec:daq}


\input daq.tex

\subsection{DCS}

\input dcs.tex


%-------------------------------------------------------------------------------
\section{Conclusion}
\label{sec:conclusion}
%-------------------------------------------------------------------------------

After three years of design and development the new \laser~calibration system was installed in USA15 in October 2014. We have tried to learn from the lessons of the former setup and our efforts have focused three critical points: optics, monitoring and electronics. A new arrangement of the optics box has been setup to ensure stability, to minimize beam pointing effects and to attenuate dependence on vibration. We move to a horizontal setup of the laser box in order to ease maintenance to minimize dust deposits on optical components.
One of the lessons learnt from \lasi~concerned the requirement to measure the \laser~signal at the output of each optical component. The number of photodiodes was thus increased to ten and a new internal calibration system had then to be setup. A challenging aspect of the project concerned the electronics. To solve the shortcomings observed with \lasi~and to benefit from recent hardware developments, it has been decided to move to a more compact setup so as to improve the timing and the transmission of the many signals that are send and received through the system.

The \lasii~setup has been used on a regular basis since October 2014. The performance of the system is compatible with expectations and will be the subject of another document that will be published shortly.

\begin{comment}
%-------------------------------------------------------------------------------
\section*{Acknowledgements}
%-------------------------------------------------------------------------------

%% Acknowledgements for papers with collision data
% Version 19-Feb-2015

% Standard acknowledgements start here
%----------------------------------------------
We thank CERN for the very successful operation of the LHC, as well as the
support staff from our institutions without whom ATLAS could not be
operated efficiently.

We acknowledge the support of ANPCyT, Argentina; YerPhI, Armenia; ARC,
Australia; BMWFW and FWF, Austria; ANAS, Azerbaijan; SSTC, Belarus; CNPq and FAPESP,
Brazil; NSERC, NRC and CFI, Canada; CERN; CONICYT, Chile; CAS, MOST and NSFC,
China; COLCIENCIAS, Colombia; MSMT CR, MPO CR and VSC CR, Czech Republic;
DNRF, DNSRC and Lundbeck Foundation, Denmark; EPLANET, ERC and NSRF, European Union;
IN2P3-CNRS, CEA-DSM/IRFU, France; GNSF, Georgia; BMBF, DFG, HGF, MPG and AvH
Foundation, Germany; GSRT and NSRF, Greece; RGC, Hong Kong SAR, China; ISF, MINERVA, GIF, I-CORE and Benoziyo Center, Israel; INFN, Italy; MEXT and JSPS, Japan; CNRST, Morocco; FOM and NWO, Netherlands; BRF and RCN, Norway; MNiSW and NCN, Poland; GRICES and FCT, Portugal; MNE/IFA, Romania; MES of Russia and ROSATOM, Russian Federation; JINR; MSTD,
Serbia; MSSR, Slovakia; ARRS and MIZ\v{S}, Slovenia; DST/NRF, South Africa;
MINECO, Spain; SRC and Wallenberg Foundation, Sweden; SER, SNSF and Cantons of
Bern and Geneva, Switzerland; NSC, Taiwan; TAEK, Turkey; STFC, the Royal
Society and Leverhulme Trust, United Kingdom; DOE and NSF, United States of
America.

The crucial computing support from all WLCG partners is acknowledged
gratefully, in particular from CERN and the ATLAS Tier-1 facilities at
TRIUMF (Canada), NDGF (Denmark, Norway, Sweden), CC-IN2P3 (France),
KIT/GridKA (Germany), INFN-CNAF (Italy), NL-T1 (Netherlands), PIC (Spain),
ASGC (Taiwan), RAL (UK) and BNL (USA) and in the Tier-2 facilities
worldwide.
%----------------------------------------------



The \texttt{atlaslatex} package contains the acknowledgements that were valid 
at the time of the release you are using.
These can be found in the \texttt{acknowledgements} subdirectory.
When your ATLAS paper or PUB/CONF note is ready to be published,
download the latest set of acknowledgements from:\\
\url{https://twiki.cern.ch/twiki/bin/view/AtlasProtected/PubComAcknowledgements}

The supporting notes for the analysis should also contain a list of contributors.
This information should usually be included in \texttt{mydocument-metadata.tex}.
The list should be printed either here or before the table of contents.
\end{comment}

%-------------------------------------------------------------------------------
\clearpage
\appendix

%\part*{Appendix A}
\input appendix.tex
%\addcontentsline{toc}{part}{Appendix}
%-------------------------------------------------------------------------------
\begin{comment}
In a paper, an appendix is used for technical details that would otherwise disturb the flow of the paper.
Such an appendix should be printed before the Bibliography.
\end{comment}
\clearpage
%-------------------------------------------------------------------------------
% If you use biblatex and either biber or bibtex to process the bibliography
% just say \printbibliography here
\printbibliography
\addcontentsline{toc}{part}{References}
% If you want to use the traditional BibTeX you need to use the syntax below.
%\bibliographystyle{bibtex/bst/atlasBibStyleWoTitle}
%\bibliography{LaserII_note,bibtex/bib/ATLAS}
%\input biblio.tex
%-------------------------------------------------------------------------------

%-------------------------------------------------------------------------------
% Print the list of contributors to the analysis
% The argument gives the fraction of the text width used for the names
%-------------------------------------------------------------------------------
\clearpage
%\PrintAtlasContribute{0.30}

\begin{comment}
%-------------------------------------------------------------------------------
\clearpage
\appendix
\part*{Auxiliary material}
\addcontentsline{toc}{part}{Auxiliary material}
%-------------------------------------------------------------------------------

In an ATLAS paper, auxiliary plots and tables that are supposed to be made public 
should be collected in an appendix that has the title \enquote{Auxiliary material}.
This appendix should be printed after the Bibliography.
At the end of the paper approval procedure, this information can be split into a separate document
-- see \texttt{atlas-auxmat.tex}.

In an ATLAS note, use the appendices to include all the technical details of your work
that are relevant for the ATLAS Collaboration only (e.g.\ dataset details, software release used).
This information should be printed after the Bibliography.

\end{comment}

\end{document}
