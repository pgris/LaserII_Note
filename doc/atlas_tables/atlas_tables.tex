%-------------------------------------------------------------------------------
% Examples on how to make tables for ATLAS documents
% Responsible: Ian Brock (Ian.Brock@cern.ch)
%-------------------------------------------------------------------------------
\documentclass[UKenglish,texlive=2013]{latex/atlasdoc}

% Standard packages that can and should be used in ATLAS papers
% Use biblatex and biber for the bibliography
\usepackage[backend=biber]{latex/atlaspackage}
\usepackage{latex/atlasbiblatex}
\usepackage{authblk}
\usepackage{verbatim}

\usepackage{latex/atlasphysics}

\usepackage{atlas_tables-defs}

% Files with references in BibTeX format
\addbibresource{../atlas_latex.bib}

\graphicspath{{../../logos/}}

%-------------------------------------------------------------------------------
% Generic document information
%-------------------------------------------------------------------------------

% Set author and title for the PDF file
\hypersetup{pdftitle={ATLAS tables guide},pdfauthor={Ian Brock}}

\AtlasTitle{Guide to formatting tables for ATLAS documents}

\author{Ian C. Brock}
\affil{University of Bonn}

\AtlasAbstract{%
  This document illustrates the preferred style for tables in ATLAS documents.
  It illustrates what the tables should look like and also provides guidelines on how to achieve this look.
  
  This document was generated using version \ATPackageVersion\ of the ATLAS \LaTeX\ package.
  The \TeX\ Live version is set to \ATTeXLiveVersion.
  It uses the option \Option{atlasstyle}, which implies that the standard ATLAS preprint style is used.
}


%-------------------------------------------------------------------------------
\begin{document} 

\maketitle

%-------------------------------------------------------------------------------
\section{General guidelines}
%-------------------------------------------------------------------------------

Tables should only contain as many lines as are needed for clarity.
Table~\ref{tab:yield:2dig} shows a good example that has been taken
from \enquote{Rounding -- ATLAS Recommendations}~\cite{atlas-rounding}.

\ifthenelse{\ATTeXLiveVersion < 2012}{%
  \sisetup{retainplus=true}
}{
  \sisetup{retain-explicit-plus=true}
}
\begin{table}[htbp]
  \centering
  \begin{tabular}{%
      l
      r@{$\,\pm\,$}r
    }
    \toprule
    {Channel} & \multicolumn{2}{c}{Selected events} \\
    \midrule
    $WW, WZ, ZZ$               & \numRF{943.045}{3}  & \numRF{94.3045}{2} \\
    QCD multijets              & \numRF{2838.39}{2}  & \numRF{1419.19}{2} \\
    $Wc\bar{c}, Wb\bar{b}, Wc$ & \numRF{31178  }{2}  & \numRF{13094.8}{2} \\
    $W$ + jets                 & \numRF{10584.5}{3}  & \numRF{4445.49}{2} \\
    Single top $Wt$            & \numRF{1699.75}{3}  & \numRF{152.977}{2} \\
    $Z$ + jets                 & \numRF{2378.42}{2}  & \numRF{998.934}{2} \\
    Single top $s$             & \numRF{297.591}{3}  & \numRF{12.4988}{2} \\
    Single top $t$             & \numRF{3936.98}{3}  & \numRF{165.353}{2} \\
    $t\bar{t}$                 & \numRF{9386.28}{3}  & \numRF{901.083}{2} \\
    \midrule
    Expected                   & \numRF{63243}{2}    & \numRF{13968.5}{2} \\
    Data                       & \multicolumn{2}{l}{\num{73062}} \\
    \bottomrule
  \end{tabular}
  \caption{Example event yields spread over several orders of
    magnitude.}
  \label{tab:yield:2dig}
\end{table}

An example of a wider and somewhat more complicated table is shown in \Tab{\ref{tab:intro:presentlimits}}.

\begin{table}[htbp]
  \centering
  \begin{tabular}{lllll}
  \toprule
  Coupling & \multicolumn{2}{c}{LEP} & \multicolumn{2}{c}{HERA} \\
  \midrule
  $\BR(t\to q\gamma)$ & \num{2.4E-2} &                       & \num{6.4E-3} & ($tu\gamma$)\\  
  $\BR(t\to qZ)$      & \num{7.8E-2} &                       & \num{49E-2} & ($tuZ$)\\
  $\BR(t\to qg)$      & \num{17E-2}  &                       & \num{13E-2} & \\
  \bottomrule
  Coupling & \multicolumn{2}{c}{Tevatron}                    & \multicolumn{2}{c}{LHC} \\
  \midrule
  $\BR(t\to q\gamma)$ & \num{3.2E-2} &                       & \multicolumn{1}{c}{---} \\
  $\BR(t\to qZ)$      & \num{3.2E-2} &                       & \num{7.0E-4}\\
  $\BR(t\to qg)$      & \num{2.0E-4} & ($tug$), $(2 \to 2)$  & \multicolumn{1}{c}{---} \\
                      & \num{3.9E-3} & ($tcg$), $(2 \to 2)$  & \multicolumn{1}{c}{---} \\
                      & \num{3.9E-4} & ($tug$), $(2 \to 1)$  & \num{5.7E-5} & ($tug$), $(2 \to 1)$\\
                      & \num{5.7E-3} & ($tcg$), $(2 \to 1)$  & \num{2.7E-4} & ($tcg$), $(2 \to 1)$\\
  \bottomrule
  \end{tabular}
  \caption[Present FCNC top quark decays experimental limits]{Present experimental limits at \SI{95}{\%} confidence level 
    on the branching fractions of the FCNC top quark decay channels established by experiments of the LEP, HERA, Tevatron and LHC accelerators. }
  \label{tab:intro:presentlimits}
\end{table}

A typical table containing Monte Carlo samples is given in \Tab{\ref{tab:mcsamples}}.

\begin{table}[htbp]
\centering
\renewcommand{\arraystretch}{1.2}
\begin{tabular}{lSlS[table-format=9.0]SS[table-format=6.0]}
\toprule
  & {$\sigma$ [\si{\pb}]} & Generator
  & \multicolumn{1}{c}{$N_{MC}$} & \multicolumn{1}{c}{$k$-factor} & \multicolumn{1}{c}{Dataset ID}\\
\midrule
$Wt$ all decays                         &  22           & \POWHEG+\PYTHIA        &   1000000  & 1.09 & 110140\\
$t$-channel (lepton+jets) top           &  18           & \POWHEG+\PYTHIA        &   5000000  & 1.05 & 110090\\
$s$-channel (lepton+jets) antitop       &  1.8          & \POWHEG+\PYTHIA        &   5000000  & 1.06 & 110091\\
$t\bar{t}$ no fully hadronic            & 114           & \POWHEG+\PYTHIA        & 100000000  & 1.12 & 117050\\
\bottomrule
\end{tabular}
\caption{Top quark event MC samples used for this analysis. The cross-section column includes $k$-factors and branching ratios. }
%The ATLAS geometry tag {\it ATLAS-GEO-16-00-00} is used and the reconstruction software was based on {\it release 16.6.3.5.1}.}
%(WARNING: update geo tag and software version)
\label{tab:mcsamples}
\end{table}

Table~\ref{tab:example1} shows the use of $\pm$ as the intercolumn
character for alignment. An alternative, as shown in
Table~\ref{tab:example2}, is to use \verb+\phantom+ to put in extra
space equal to the width of a number if you have different numbers of
decimal places in the table.

The \texttt{booktabs} package provides the macros 
\Macro{toprule}, \Macro{midrule}, \Macro{bottomrule} which are to be preferred over \Macro{hline},
as, among other things, they introduce some extra spacing around the lines, which is useful.

\begin{table}[htbp]
  \centering
  \begin{tabular}{l S[table-format=2.1]@{$\;\pm\;$}S[table-format=1.1]@{\,}s
    S[table-format=3.1]@{\,}s}
    \toprule
    Category            & \multicolumn{3}{c}{$\mu$}&\multicolumn{2}{c}{$e$}\\ 
    \midrule
    $b \to \ell$        &     65.2 & 0.4 & \%   &  79.3 & \% \\
    $b \to c \to \ell$  &      7.8 & 0.3 & \%   &   5.4 & \% \\
    Total               &     73.0 & 0.2 & \%   &   9.1 & \% \\ 
    \bottomrule
  \end{tabular}
  \caption[Monte Carlo purities in the single lepton sample]{%
    Monte Carlo estimates of the fraction of each process in the single
    lepton data sample. This table uses ``S'' format from \texttt{siunitx} and
    ``\texttt{$\,\pm\,$}'' as the intercolumn separator.}
  \label{tab:example1}
\end{table}


\begin{table}[htbp]
  \centering
  \begin{tabular}{lcc}
    \toprule
    Category            & \multicolumn{1}{c}{$\mu$}&\multicolumn{1}{c}{$e$}\\ 
    \midrule
    $b \to \ell$        & $    65.2 \pm 0.4\,\%$   &     79.3\,\% \\
    $b \to c \to \ell$  & $\pho 7.8 \pm 0.3\,\%$   & \pho 5.4\,\% \\
    Total               & $    73.0 \pm 0.2\,\%$   & \pho 9.1\,\% \\ 
    \bottomrule
  \end{tabular}
  \caption{Monte Carlo estimates of the fraction of each process in
    the single lepton data sample. 
    This table uses \texttt{\textbackslash phantom}.}
  \label{tab:example2}
\end{table}

%-------------------------------------------------------------------------------
\section{\LaTeX\ packages for table}
%-------------------------------------------------------------------------------

The \LaTeX\ package \Package{booktabs} gives a number of guidelines on how tables should be formatted.
These are followed to a large extent in this document.
The following packages related to tables are included by default when you load the package \Package{atlaspackage}:
\begin{description}
\item[booktabs] useful tools for formatting tables;
\item[siunitx] tools for rounding and also for helping to format and align numbers in tables;
\end{description}
Further packages related to the formatting of tables are:
\begin{description}
\item[xtab] the most modern package for tables that spread over more than one page;
\item[longtable] an alternative package for long tables;
\item[supertabular] yet another alternative package for long tables;
\item[dcolumn] can be used as an alternative to \Package{siunitx} to align numbers in tables.
\end{description}
\Package{xtab} is included if you load \Package{atlaspackage} with the option \Option{maximal}.
You may also need to rotate a big table. 
The \Package{rotating} package can be used for this.

In order to shorten commands when doing rounding in tables, it is useful to define a few extra macros.
Typical definitions can be found in the file \File{atlas\_tables-defs.sty}.


%-------------------------------------------------------------------------------
\section{Tables and rounding}
%-------------------------------------------------------------------------------

Further examples of tables can be found in the note discussing the ATLAS recommendation on rounding~\cite{atlas-rounding}.
A selection of those tables are also reproduced here.
The \LaTeX\ code for the examples given below can be found in 
Appendix~\ref{sec:raw-data}.

The tables shown earlier in this document were also created with \textsf{siunitx}.
A few more examples of how to steer the formatting are given here.
Table~\ref{tab:rounding:xsect} compares two different approaches
to how this can be done  in \textsf{siunitx}, even for asymmetric errors.  Note that although these
tables look almost identical, the syntax used to create them is different (see Appendix~\ref{sec:raw-data}).
While the form may appear to be a bit clumsy at first, it is easy enough to get a
program to write out the lines. In the left-hand table
\Macro{numRP} is used in column 3, while the full syntax of \Macro{num} 
in shown in column 4 for illustration purposes only.  The syntax
to change the precision of a single number is shown in the first line of
the left-hand part of the table. This is seen to be rather
trivial, but the alignment on the decimal point is now no longer
perfect. While this is probably OK for internal notes etc., papers
(should) have more stringent requirements. Another way of achieving
the same thing and avoiding the use of \textsf{round-mode} and
\textsf{round-precision} is shown in the code for the right-hand table. Note the
use of options for the \textsf{S} format and the use of \Macro{num} enclosed
in braces to format the row that requires a different precision.

\begin{table}[htbp]
  \centering
  \renewcommand{\arraystretch}{1.4}
  \sisetup{retain-explicit-plus}
  \sisetup{round-mode = places}
  \begin{tabular}{%
      S@{\,:\,}S
      r@{\,}@{$\pm$}@{\,}l@{\,}l
       }
    \toprule
    \multicolumn{2}{c}{\etajet} & \multicolumn{3}{c}{\diffetab} \\
    \multicolumn{2}{c}{} & \multicolumn{3}{c}{[\si{\pico\barn}]} \\
    \midrule
    {\num{-1.6}} & -1.1 & \numRP{0.574}{3} & \num[round-precision=3]{0.094} & $^{\numRP{+0.035}{3}}_{\numRP{-0.031}{3}}$ \\
    {\num{-1.1}} & -0.8 & \numRP{1.213}{2} & \num[round-precision=2]{0.211} & $^{\numRP{+0.162}{2}}_{\numRP{-0.162}{2}}$ \\
    {\num{-0.8}} & -0.5 & \numRP{2.141}{2} & \num[round-precision=2]{0.219} & $^{\numRP{+0.223}{2}}_{\numRP{-0.123}{2}}$ \\
    {\num{-0.5}} & -0.2 & \numRP{2.326}{2} & \num[round-precision=2]{0.210} & $^{\numRP{+0.284}{2}}_{\numRP{-0.214}{2}}$ \\
    {\num{-0.2}} & +0.1 & \numRP{2.641}{2} & \num[round-precision=2]{0.220} & $^{\numRP{+0.283}{2}}_{\numRP{-0.233}{2}}$ \\
    {\num{+0.1}} & +0.5 & \numRP{3.160}{2} & \num[round-precision=2]{0.211} & $^{\numRP{+0.232}{2}}_{\numRP{-0.172}{2}}$ \\
    {\num{+0.5}} & +1.4 & \numRP{2.881}{2} & \num[round-precision=2]{0.154} & $^{\numRP{+0.201}{2}}_{\numRP{-0.301}{2}}$ \\
    \bottomrule
  \end{tabular}
  \quad
  \sisetup{round-mode = places, round-precision = 2}
  \begin{tabular}{%
      S[table-format=3.2, table-number-alignment = right]@{\,:\,}S
      S[round-mode = places, round-precision = 2,
      table-format = 1.3, table-number-alignment = right]
      @{$\,\pm\,$}
      S[round-mode = places, round-precision = 2,
      table-format = 1.3, table-number-alignment = left]
      @{\,}l
       }
    \toprule
    \multicolumn{2}{c}{\etajet} & \multicolumn{3}{c}{\diffetab} \\
    \multicolumn{2}{c}{} & \multicolumn{3}{c}{[\si{\pico\barn}]} \\
    \midrule
    -1.6 & -1.1 & {\numRP{0.574}{3}} & {\numRP{0.094}{3}} & $^{\numRP{+0.035}{3}}_{\numRP{-0.031}{3}}$ \\
    -1.1 & -0.8 & 1.213 & 0.211 & $^{\num{+0.162}}_{\num{-0.162}}$ \\
    -0.8 & -0.5 & 2.141 & 0.219 & $^{\num{+0.223}}_{\num{-0.123}}$ \\
    -0.5 & -0.2 & 2.326 & 0.210 & $^{\num{+0.284}}_{\num{-0.214}}$ \\
    -0.2 & +0.1 & 2.641 & 0.220 & $^{\num{+0.283}}_{\num{-0.233}}$ \\
    +0.1 & +0.5 & 3.160 & 0.211 & $^{\num{+0.232}}_{\num{-0.172}}$ \\
    +0.5 & +1.4 & 2.881 & 0.154 & $^{\num{+0.201}}_{\num{-0.301}}$ \\
    \bottomrule
  \end{tabular}
  \caption{A selection of cross-section measurements. Note that
    for numbers with asymmetric errors, the option 
    \texttt{\Macro{sisetup}\{retain-explicit-plus\}} is used to stop 
    \textsf{siunitx} from dropping the plus signs on the positive
    errors. (although these tables look almost identical, the syntax used to 
    create them is different - see Appendix~\ref{sec:raw-data}).}
  \label{tab:rounding:xsect}
\end{table}

Cross-sections vs.\ $\eta$ are usually not so difficult to
format, as the magnitudes of the numbers do not change much from one
bin to the next. The situation is different for cross-sections as a
function of \ET\ or $x$. Tables~\ref{tab:xsect-ET} and
\ref{tab:xsect-x} show examples of such tables.

\begin{table}[htbp]
  \centering
  \renewcommand{\arraystretch}{1.4}
  \subfloat[No special formatting and
  \texttt{round-mode=figures}. This is the starting point for more
  refined formatting.]{%
    \label{tab:xsect-ET1}%
    %Charm differential cross sections d sigma / dY in bins of Et
\typeout{ATTeXLiveVersion is [\ATTeXLiveVersion]}
\ifthenelse{\ATTeXLiveVersion < 2012}{%
  \sisetup{round-mode=figures, round-precision=2,
  retainplus=true, group-integer-digits=true, group-four-digits=true}
}{%
  \sisetup{round-mode=figures, round-precision=2,
  retain-explicit-plus=true, group-digits=integer, group-minimum-digits=4}
}
\begin{tabular}{%
    S[table-format=2.0, table-number-alignment=right,
    round-mode=places, round-precision=0]@{$\,:\,$}
    S[table-format=2.0, table-number-alignment=left,
    round-mode=places, round-precision=0]
    S[table-format=4.2, table-number-alignment=right,
    round-mode=figures, round-precision=3]@{$\,\pm\,$}
    S[table-format=3.2, table-number-alignment=right,
    round-mode=figures, round-precision=2]@{$\,$}l}
  \toprule
  \multicolumn{2}{c}{\ET} &
  \multicolumn{3}{c}{$\dif\sigma / \dif\ET$}\\
  \multicolumn{2}{c}{\mbox{}} & \multicolumn{3}{c}{[\si{\pico\barn\per\GeV}]}\\
  \midrule
 4.2 & 8.0  & 3634.06 & 114.491  & \numpmerr{+201.404 }{-181.511}{2}  \\
 8.0 & 11.0 & 719.458 & 21.9334  & \numpmerr{+43.3087 }{-39.7824}{2}  \\
11.0 & 14.0 & 214.572 & 9.71991  & \numpmerr{+20.5413 }{-19.6464}{2}  \\
14.0 & 17.0 & 85.7584 & 6.03401  & \numpmerr{+10.0875 }{-8.99952}{2}  \\
17.0 & 20.0 & 35.4095 & 3.91591  & \numpmerr{+5.5349  }{-5.41347}{2}  \\
20.0 & 25.0 & 14.1253 & 2.72552  & \numpmerr{+3.46528 }{-3.22476}{2}  \\
25.0 & 35.0 & 2.37786 & 0.968562 & \numpmerr{+0.849647}{-0.855525}{2} \\
  \bottomrule
\end{tabular}
}
  \qquad
  \subfloat[Numbers adjusted according to the recommendations. \texttt{round-mode=places}
  is used for asymmetric errors (except the first row). Some judicious
  use of \Macro{phantom} is applied to get improved, but not yet perfect, alignment.]{%
    \label{tab:xsect-ET2}%
    %Charm differential cross sections d sigma / dY in bins of Et\
\typeout{ATTeXLiveVersion is [\ATTeXLiveVersion]}
\ifthenelse{\ATTeXLiveVersion < 2012}{%
  \sisetup{round-mode=figures, round-precision=2,
  retainplus=true, group-integer-digits=true, group-four-digits=true}
}{%
  \sisetup{round-mode=figures, round-precision=2,
  retain-explicit-plus=true, group-digits=integer, group-minimum-digits=4}
}
\begin{tabular}{%
    S[table-format=2.0, table-number-alignment=right,
    round-mode=places, round-precision=0]@{$\,:\,$}
    S[table-format=2.0, table-number-alignment=left,
    round-mode=places, round-precision=0]
    S[table-format=4.1, table-alignment=right,
    round-mode=figures, round-precision=3]@{$\,\pm\,$}
    S[table-format=3.1, table-alignment=right,
    round-mode=figures, round-precision=2]@{$\,$}r}
  \toprule
  \multicolumn{2}{c}{\ET} &
  \multicolumn{3}{c}{$\dif\sigma / \dif\ET$}\\
  \multicolumn{2}{c}{[\si{\GeV}]} & \multicolumn{3}{c}{[\si{\pico\barn\per\GeV}]}\\
  \midrule
 4.2 & 8.0  & 3634.06                    & 114.491                    & \numpmRF{+201.404 }{-181.511 }{2}  \\
 8.0 & 11.0 & 719.458                    & 21.9334                    & \numpmerr{+43.3087 }{-39.7824 }{0} \\
11.0 & 14.0 & {\numRF{214.572}{2}\phdo}  & {\numRF{9.71991}{1}\phdo}  & \numpmerr{+20.5413 }{-19.6464 }{0} \\
14.0 & 17.0 & {\numRF{85.7584}{2}\phdo}  & {\numRF{6.03401}{1}\phdo}  & \numpmerr{+10.0875 }{-8.99952 }{0} \\
17.0 & 20.0 & {\numRF{35.4095}{3}}       & {\numRF{3.91591}{2}}       & \numpmerr{+5.5349  }{-5.41347 }{1} \\
20.0 & 25.0 & 14.1253                    & 2.72552                    & \numpmerr{+3.46528 }{-3.22476 }{1} \\
25.0 & 35.0 & {\numRF{2.37786}{2}}       & {\numRF{0.968562}{1}}     & \numpmerr{+0.849647}{-0.855525}{1}  \\
  \bottomrule
\end{tabular}
}
  \caption{Cross-section vs.\ $E_{T}$.}
  \label{tab:xsect-ET}
\end{table}

\texttt{round-mode=figures} is in general best for cross-sections and
their errors. A precision of 2 digits for the uncertainties is a good
starting point, but will then have to be reduced to 1 digit in some cases. 
For the cross-section values, more digits (typically 3) probably have to
be specified and the precision of some values will again have to be
adjusted by hand. In Table~\ref{tab:xsect-ET2} some of the rounding 
is adjusted by hand so that the numbers conform to the
rules. For the asymmetric errors, \texttt{round-mode=places} is used
and the precision of each asymmetric uncertainty is then set by hand. 
This works well if the cross-sections should all be shown with decimal
points, but does not work if used to round a number such as
\num{182}. Hence the first row uses \texttt{round-mode=figures}. Even
with the tools offered by \Macro{siunitx} getting things exactly right
is non-trivial.

\begin{table}[htb]
  \centering
  \renewcommand{\arraystretch}{1.4}
  \ifthenelse{\ATTeXLiveVersion < 2012}{%
    \textbf{This table does not appear here, as it uses features that were only 
    properly implemented for \TeX\ Live 2012  and later.}
  }{%
    \subfloat[No special formatting or rounding. Option
      \textsf{scientific-notation=fixed} used.]{%
      \label{tab:xsect-x1}%
      %Charm differential cross sections d sigma / dY in bins of xda
\typeout{ATTeXLiveVersion is [\ATTeXLiveVersion]}
\ifthenelse{\ATTeXLiveVersion < 2012}{%
  \sisetup{round-mode=figures, round-precision=2,
  retainplus=true, group-integer-digits=true, group-four-digits=true}
}{%
  \sisetup{round-mode=figures, round-precision=2,
  retain-explicit-plus=true, group-digits=integer, group-minimum-digits=4}
}
\sisetup{scientific-notation=fixed, fixed-exponent=0}
\begin{tabular}{%
    S[table-format=1.5, table-number-alignment=right,
    round-mode=figures, round-precision=1]@{$\,:\,$}
    S[table-format=1.5, table-number-alignment=left,
    round-mode=figures, round-precision=1]
    S[table-format=8.0, table-number-alignment=right,
    round-mode=figures, round-precision=3]@{$\,\pm\,$}
    S[table-format=6.0, table-number-alignment=right,
    round-mode=figures, round-precision=2]@{$\,$}r}
  \toprule
  \multicolumn{2}{c}{$x$} &
  \multicolumn{3}{c}{$\dif\sigma / \dif x$}\\
  \multicolumn{2}{c}{\mbox{}} & \multicolumn{3}{c}{[\si{\pico\barn}]}\\
  \midrule
0.00008 & 0.00020 & 1.08474e+07 & 867945  & \numpmerr{+761437 }{-647690 }{2}  \\
0.00020 & 0.00060 & 1.08385e+07 & 388976  & \numpmerr{+567443 }{-441257 }{2}  \\
0.00060 & 0.00160 & 4.974e+06   & 135404  & \numpmerr{+256385 }{-233376 }{2}  \\
0.00160 & 0.00500 & 1.21664e+06 & 31162.1 & \numpmerr{+68948.1}{-62459.6}{2} \\
0.00500 & 0.01000 & 256870      & 12232.7 & \numpmerr{+18363.7}{-16463.7}{2} \\
0.01000 & 0.10000 & 10652.6     &  791.21 & \numpmerr{+913.118}{-815.675}{2} \\
  \bottomrule
\end{tabular}

    }
    \qquad
    \subfloat[Several fixes including rescaled cross-section. Quite a
      lot of \Macro{phantom} commands are applied to get alignment correct.]{%
      \label{tab:xsect-x2}%
      %Charm differential cross sections d sigma / dY in bins of xda
\typeout{ATTeXLiveVersion is [\ATTeXLiveVersion]}
\ifthenelse{\ATTeXLiveVersion < 2012}{%
  \sisetup{round-mode=figures, round-precision=2,
  retainplus=true, group-integer-digits=true, group-four-digits=true}
}{%
  \sisetup{round-mode=figures, round-precision=2,
  retain-explicit-plus=true, group-digits=integer, group-minimum-digits=4}
}
%\sisetup{scientific-notation=false}
\sisetup{scientific-notation=fixed, fixed-exponent=0}
\begin{tabular}{%
    S[table-format=1.5, table-number-alignment=right,
    round-mode=figures, round-precision=1]@{$\,:\,$}
    S[table-format=1.5, table-number-alignment=left,
    round-mode=figures, round-precision=1]
    S[table-format=5.1, table-alignment=right,
    round-mode=figures, round-precision=4]@{$\,\pm\,$}
    S[table-format=3.1, table-alignment=right,
    round-mode=figures, round-precision=2]@{$\,$}r}
  \toprule
  \multicolumn{2}{c}{$x$} &
  \multicolumn{3}{c}{$\dif\sigma / \dif x$}\\
  \multicolumn{2}{c}{\mbox{}} & \multicolumn{3}{c}{[\si{\nano\barn}]}\\
  \midrule
0.00008 & 0.00020                 & {\numRF{1.08474e+04}{2}\phdo}  & {\numRF{867945e-3}{1}\phdo}  & \numpmerr{+761437 e-3}{-647690 e-3}{1} \\
0.00020 & 0.00060                 & {\numRF{1.08385e+04}{3}\phdo}  & {\numRF{388976e-3}{1}\phdo}  & \numpmerr{+567443 e-3}{-441257 e-3}{1} \\
0.00060 & {\numRF{0.0016}{2}\pho} & {\numRF{4.974e+03}{3}\phdo}    & 135404e-3                    & \numpmerr{+256385 e-3}{-233376 e-3}{2} \\
{\numRF{0.0016}{2}\pho} & 0.00500 & {\numRF{1.21664e+03}{4}\phdo}  & 31162.1e-3                   & \numpmerr{+68948.1e-3}{-62459.6e-3}{2} \\
0.00500 & 0.01000                 & {\numRF{256870e-03}{3}\phdo}   & 12232.7e-3                   & \numpmerr{+18363.7e-3}{-16463.7e-3}{2} \\
0.01000 & 0.10000                 & {\numRF{10652.6e-03}{3}}       & {\numRF{791.21e-3}{1}}       & \numpmerr{+913.118e-3}{-815.675e-3}{1} \\
  \bottomrule
\end{tabular}

    }
  }
  \caption{Cross-section vs.\ $x$.}
  \label{tab:xsect-x}
\end{table}

Table~\ref{tab:xsect-x} is probably the most challenging to format
correctly, as the bin boundaries also vary by several orders of
magnitude. Table~\ref{tab:xsect-x1} gives the numbers with the option
\textsf{scientific-notation=fixed} to illustrate the problem of what
the table would look like if the cross-sections are output in
\si{\pb}.  In Table~\ref{tab:xsect-x2}, the exponential format of
numbers is used to rescale the cross-section from \si{\pb} to
\si{\nb}.  \Macro{phantom} had to be used in more places than we
really like in order to get the final alignment
correct. Investigations are ongoing to see if this can be improved.

%-------------------------------------------------------------------------------
\section*{History}
%-------------------------------------------------------------------------------

\begin{description}
\item[2014-11-25: Ian Brock] First version of the document released.
\end{description}

%-------------------------------------------------------------------------------
% Print bibliography using biblatex
\printbibliography
%-------------------------------------------------------------------------------
% Old style bibtex bibliography
% \bibliographystyle{../../bibtex/bst/atlasBibStyleWithTitle}
% \bibliography{atlas_bibtex,atlas_biblatex}

\appendix
\section{\LaTeX\ code for tables}
\label{sec:raw-data}

This appendix gives the \LaTeX\ code including the raw data used for 
Tables~\ref{tab:rounding:xsect}, 
\ref{tab:xsect-ET} and \ref{tab:xsect-x}.
These files for Tables~\ref{tab:xsect-ET} and \ref{tab:xsect-x} can also be 
found in the \texttt{doc/atlas\_tables} directory of the \texttt{atlaslatex} package.
The options given here correspond to those that are need for \TeX\ Live 2012 and later.
See the code for the appropriate options for earlier versions.

\subsection{Table~\protect\ref{tab:rounding:xsect}}
\begin{verbatim}
\begin{table}[htbp]
\centering
\renewcommand{\arraystretch}{1.4}
\sisetup{retain-explicit-plus}
\sisetup{round-mode = places}
\begin{tabular}{%
S@{\,:\,}S
r@{\,}@{$\pm$}@{\,}l@{\,}l
}
\toprule
\multicolumn{2}{c}{\etajet} & \multicolumn{3}{c}{\diffetab} \\
\multicolumn{2}{c}{} & \multicolumn{3}{c}{[\si{\pico\barn}]} \\
\midrule
{\num{-1.6}} & -1.1 & \numRP{0.574}{3} & \num[round-precision=3]{0.094} & 
$^{\numRP{+0.035}{3}}_{\numRP{-0.031}{3}}$ \\
{\num{-1.1}} & -0.8 & \numRP{1.213}{2} & \num[round-precision=2]{0.211} & 
$^{\numRP{+0.162}{2}}_{\numRP{-0.162}{2}}$ \\
{\num{-0.8}} & -0.5 & \numRP{2.141}{2} & \num[round-precision=2]{0.219} & 
$^{\numRP{+0.223}{2}}_{\numRP{-0.123}{2}}$ \\
{\num{-0.5}} & -0.2 & \numRP{2.326}{2} & \num[round-precision=2]{0.210} & 
$^{\numRP{+0.284}{2}}_{\numRP{-0.214}{2}}$ \\
{\num{-0.2}} & +0.1 & \numRP{2.641}{2} & \num[round-precision=2]{0.220} & 
$^{\numRP{+0.283}{2}}_{\numRP{-0.233}{2}}$ \\
{\num{+0.1}} & +0.5 & \numRP{3.160}{2} & \num[round-precision=2]{0.211} & 
$^{\numRP{+0.232}{2}}_{\numRP{-0.172}{2}}$ \\
{\num{+0.5}} & +1.4 & \numRP{2.881}{2} & \num[round-precision=2]{0.154} & 
$^{\numRP{+0.201}{2}}_{\numRP{-0.301}{2}}$ \\
\bottomrule
\end{tabular}
%
\quad
%
\sisetup{round-mode = places, round-precision = 2}
\begin{tabular}{%
S[table-format=3.2, table-number-alignment = right]@{\,:\,}S
S[round-mode = places, round-precision = 2,
table-format = 1.3, table-number-alignment = right]
@{$\,\pm\,$}
S[round-mode = places, round-precision = 2,
table-format = 1.3, table-number-alignment = left]
@{\,}l
}
\toprule
\multicolumn{2}{c}{\etajet} & \multicolumn{3}{c}{\diffetab} \\
\multicolumn{2}{c}{} & \multicolumn{3}{c}{[\si{\pico\barn}]} \\
\midrule
-1.6 & -1.1 & {\numRP{0.574}{3}} & {\numRP{0.094}{3}} & 
$^{\numRP{+0.035}{3}}_{\numRP{-0.031}{3}}$ \\
-1.1 & -0.8 & 1.213 & 0.211 & $^{\num{+0.162}}_{\num{-0.162}}$ \\
-0.8 & -0.5 & 2.141 & 0.219 & $^{\num{+0.223}}_{\num{-0.123}}$ \\
-0.5 & -0.2 & 2.326 & 0.210 & $^{\num{+0.284}}_{\num{-0.214}}$ \\
-0.2 & +0.1 & 2.641 & 0.220 & $^{\num{+0.283}}_{\num{-0.233}}$ \\
+0.1 & +0.5 & 3.160 & 0.211 & $^{\num{+0.232}}_{\num{-0.172}}$ \\
+0.5 & +1.4 & 2.881 & 0.154 & $^{\num{+0.201}}_{\num{-0.301}}$ \\
\bottomrule
\end{tabular}
%
\caption{A selection of cross-section measurements! Note the
use of \Macro{sisetup} to keep the plus signs on the positive
errors.}
\label{tab:rounding:xsect}
\end{table}
\end{verbatim}

\subsection{Table~\protect\ref{tab:xsect-ET}}
The files are: \texttt{cross\_sections\_charm-ET1.tex} and 
\texttt{cross\_sections\_charm-ET2.tex}:
{\scriptsize
  \verbatiminput{cross_sections_charm-ET1.tex}
}
{\scriptsize
  \verbatiminput{cross_sections_charm-ET2.tex}
}

\subsection{Table~\protect\ref{tab:xsect-x}}

The files are: \texttt{cross\_sections\_charm-x1.tex} and 
\texttt{cross\_sections\_charm-x2.tex}:
{\scriptsize
  \verbatiminput{cross_sections_charm-x1.tex}
}
{\tiny
  \verbatiminput{cross_sections_charm-x2.tex}
}

\end{document}
